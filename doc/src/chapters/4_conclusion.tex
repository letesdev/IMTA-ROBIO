\section{Conclusion}
Dans ce travail, nous avons étudié le sens électrique \textit{bio-inspiré} de certains poissons, en particulier le mode de perception active, ou électrolocation active, utilisé notamment pour la navigation. Nous avons développé une loi de commande qui utilise le modèle d'électrolocation \og méthode des réflexions \fg{} pour simuler le mouvement d'un poisson dans un aquarium qui évite toute sorte d'obstacles (objets isolants, conducteurs et murs): il s'agit d'une loi de contrôle (simple) proportionnelle aux courants perçus par les électrodes du robot qui agit sur la vitesse angulaire du robot. 

Par la suite, nous pourrions étudier la cohabitation des poissons électriques en groupe, traité dans \cite{BENACHENHOU2014}, ou porter notre loi de commande à un environnement réel. 